\documentclass[12pt]{article}
\usepackage{setspace}
\usepackage{fullpage}
\usepackage{graphicx}
\usepackage{subfigure}
\usepackage{yhmath,wasysym}
\setlength{\parindent}{0em}
\setlength{\parskip}{0.85em}
\setstretch{1.2}
\def\bmath#1{\mbox{\boldmath$#1$}}

\usepackage{color}
\definecolor{darkblue}{rgb}{0,0,0.4}
\usepackage[colorlinks=true,citecolor=black,linkcolor=darkblue,
urlcolor=darkblue,bookmarks=false,bookmarksopen=false,
pdfpagemode=None,pdfstartview=FitH]{hyperref}
\newcommand*{\DOT}{.}

% \title{Notes on Elasticity} %
% \author{Ju Li, Department of Materials Science and Engineering, %
% University of Pennsylvania, Philadelphia, PA 19104, USA} %
% \date{\today} %

\begin{document}
\bibliographystyle{unsrt}

\newcommand{\URL}[1]{\href{#1}{#1}} 

\renewcommand\refname{References \\ ({\small
\URL{http://mt.seas.upenn.edu/Stuff/e/Notes/Paper/}
})}

% \maketitle %
\centerline{\LARGE\bf Notes on Transformation Elasticity} 

\centerline{\large\bf Ju Li, University of Pennsylvania, Sept. 27, 2008}

\section{Inhomogeneous Elasticity Solver}

Given an original supercell ${\bf x}={\bf s}{\bf H}_0$,
${\bf s}\in[0,1)$, we would like to solve the following problem:
\begin{equation}
 F^{\rm el}[{\bf H}, \bmath{\epsilon}^0({\bf x})] \;\equiv\; 
\min_{{\bf u}({\bf x})} 
F^{\rm el}[{\bf u}({\bf x}) | {\bf H}, \bmath{\epsilon}^0({\bf x})]
\end{equation}
\begin{equation}
F^{\rm el}[{\bf u}({\bf x}) | {\bf H}, \bmath{\epsilon}^0({\bf x})] \equiv
 \frac{1}{2}\int d^3{\bf x} c_{ijpq}({\bf x})
 ( \epsilon_{ij}({\bf x}) - \epsilon^0_{ij}({\bf x}) ) 
 ( \epsilon_{pq}({\bf x}) - \epsilon^0_{pq}({\bf x}) )
 \label{ElasticEnergyDefinition}
\end{equation}
where ${\bf u}({\bf x})\equiv {\bf x}^\prime - {\bf x}$, the
difference between the new position ${\bf x}^\prime$ and the old
position ${\bf x}$, and 
\begin{equation}
 \epsilon_{ij}({\bf x}) \;\equiv\; \frac{u_{i,j}+u_{j,i}}{2}.
 \label{LinearElasticStrainDefinition}
\end{equation}
${\bf H}$ is the new supercell: ${\bf H} = {\bf H}_0
({\bf I}+\bar{\bmath{\epsilon}})$.  Global rotation of {\bf H} does
not matter anyhow to (\ref{ElasticEnergyDefinition}) to first order in
the global rotation angle, and is therefore ignored.  Note that the
untransformed material in ${\bf H}_0$ does not have to be stress free.
The entire role of ${\bf H}_0$ in the above is to provide a reference
grid.

(\ref{ElasticEnergyDefinition}) is motivated by the following idea
experiment.  One first cuts up the {\em untransformed supercell} ${\bf
H}_0$ into many blocks $d^3{\bf x}$.  Then imagine for instance the
temperature is raised, and phase transformation / plasticity may
induce some blocks to transform to a new state.  Each block, if left
alone (stress free), would like to transform to a new strain state
$\bmath{\epsilon}^0({\bf x})$. Additional local rotation ${\bf R}({\bf x})$ of
the block does not matter to the internal Helmholtz free energy of
{\em this} block, but needs to be globally optimized since ${\bf
R}({\bf x})$ must be globally consistent with ${\bf u}({\bf x})$.

Because of the periodic condition, there must be 
\begin{equation}
 {\bf u}({\bf x}+{\bf
h}_0)-{\bf u}({\bf x}) \;=\; {\bf h}_0\bar{\bmath{\epsilon}}
\end{equation}
where ${\bf
h}_0$ is one of the ${\bf H}_0$ edge vectors.  So
\begin{equation}
 \int_0^{{\bf h}_0} d{\bf x}^\prime\cdot 
\frac{d{\bf u}({\bf x}+{\bf x}^\prime)}{d{\bf x}^\prime} 
 \;=\; 
 {\bf h}_0\bar{\bmath{\epsilon}}  \;\;\rightarrow\;\;
 \int d^3{\bf x} \frac{d{\bf u}}{d{\bf x}} =  \det|{\bf H}_0| 
 \bar{\bmath{\epsilon}}.
\end{equation}

Note that $\{\epsilon_{ij}({\bf x})\}$, because of
(\ref{LinearElasticStrainDefinition}), need to satisfy compatibility
constraints, which means the $\{\epsilon_{ij}({\bf x})\}$ fields are
not independent fields in the variational functional ($\{u_i({\bf
x})\}$ fields are). On the other hand, there is no such constraint on
the stress-free strain fields $\{\epsilon^0_{ij}({\bf x})\}$, which
are ``given'' in the elastic constant minimization problem.

The functional to be minimized in (\ref{ElasticEnergyDefinition})
represents a quadratic expansion approximation of the Helmholtz free
energy \cite{LiPhDThesis00} around the {\em freely transformed} block.
As such, there should be a conversion factor $\det|d^3{\bf
x}^\prime|/\det|d^3{\bf x}|$ as well tensor rotation using ${\bf
R}({\bf x})$ to convert the isothermal elastic constant of the
transformed material to $c_{ijpq}({\bf x})$ based on the original
volume and observation coordinates.  However, this effect is higher
order, same as the higher-order terms ignored in
(\ref{LinearElasticStrainDefinition}).

Unlike the more general nonlinear formulation of \cite{Jagla07}, the
merit of the quadratic expansion is that
(\ref{ElasticEnergyDefinition}) is quadratic in ${\bf u}({\bf x})$,
whose minimization (in principle at least) entertains a close-formed
solution in the form of a matrix inverse, after real-space
discretization of ${\bf u}({\bf x})$ and conversion of $\nabla^2$-like
operators. We have stress equilibrium equation in structurally
inhomogeneous and elastically inhomogeneous material:
\begin{equation}
 ( c_{ijpq}({\bf x}) 
  ( u_{p,q}({\bf x}) - \epsilon^0_{pq}({\bf x}) ))_{,j} \;=\; 0, 
  \;\;\;\;
  \forall i=1..3
 \label{StressEquilibrium}
\end{equation}


\subsection{Homogeneous Special Case}
\label{sec:HomogeneousSpecialCase}

If the system is elastically homogeneous \cite{WangJK02}, 
$c_{ijpq}({\bf x})=c_{ijpq}^0$, there is
translational symmetry in the problem:
\begin{equation}
 c_{ijpq}^0 ( 
  u_{p,q}({\bf x}) - \epsilon^0_{pq}({\bf x}) )_{,j} \;=\; 0, 
  \;\;\;\;
  \forall i=1..3
\end{equation}
and the inverse can be done in the Fourier space on a ${\bf
k}$-by-${\bf k}$ basis.  We first note that $u_p({\bf x})$ can be
decomposed into a secularly growing component in ${\bf x}$, plus a
periodic component:
\begin{equation}
 u_p({\bf x}) \;\equiv\; {\bf x}\bar{\bmath{\epsilon}} + \tilde{u}_p({\bf x})
\end{equation}
Then stress equilibrium requires that in {\bf k}-space:
\begin{equation}
  -c_{ijpq}^0 k_qk_j \tilde{u}_p({\bf k}) \;=\; 
  ic_{ijpq}^0\epsilon^0_{pq}({\bf k}) k_j
\end{equation}
where 
\begin{equation}
 \tilde{u}_p({\bf k}) \;=\; 
 \int d^3{\bf x} \tilde{u}_p({\bf x})e^{-i{\bf k}\cdot {\bf x}},
 \;\;\;\;
 \tilde{u}_p({\bf x}) \;=\; \frac{1}{\det|{\bf H}_0|}
 \sum_{{\bf k}} \tilde{u}_p({\bf k})e^{i{\bf k}\cdot {\bf x}},
\end{equation}
and similarly $\epsilon^0_{pq}({\bf
k})\leftrightarrow\epsilon^0_{pq}({\bf x})$.  If we define 
symmetric matrix ${\bf C}(\hat{\bf k})$
\cite{WangJK02}
\begin{equation}
 C_{ip}(\hat{\bf k}) \;\equiv\; c_{ijpq}^0 \hat{k}_q\hat{k}_j, \;\;\;\;
 \hat{\bf k}\equiv \frac{{\bf k}}{|{\bf k}|},
 \label{C_Definition}
\end{equation}
the inverse matrix is also symmetric: ${\bf \Omega}(\hat{\bf k})\equiv
{\bf C}^{-1}(\hat{\bf k})$.  Let us also define strain-free stress:
\begin{equation}
 \sigma^0_{ij}({\bf k}) \;\equiv\; c_{ijpq}^0\epsilon^0_{pq}({\bf k}), \;\;\;
 \sigma^0_{ij}({\bf x}) \;\equiv\; c_{ijpq}^0\epsilon^0_{pq}({\bf x})
\end{equation}
and $\tilde{u}_p({\bf k})$ is obtained
explicitly as 
\begin{equation}
 \tilde{u}_p({\bf k}) \;=\; 
 \frac{\Omega_{pi}(\hat{\bf k})\sigma^0_{ij}({\bf k}) k_j}
 {i|{\bf k}|^2}.
 \label{DisplacementHomogeneousSolution}
\end{equation}
Since $\sigma^0_{ij}({\bf k}) k_j$ represents the divergence of
stress, or net force, $\frac{\Omega_{pi}(\hat{\bf k})}{i|{\bf k}|^2}$
is just the infinite-space Green's function relating force to
displacement in this translationally invariant system.  This Green's
function is short-ranged in reciprocal space (in fact ${\bf
k}$-by-${\bf k}$ local), but long-ranged in real space.  Thus it is
advantageous to solve homogeneous-material problems in reciprocal
space, which is the spectral solution method.

The strain field that corresponds to
(\ref{DisplacementHomogeneousSolution}) displacement field is
\begin{equation}
 \tilde{\epsilon}_{pq}({\bf k}) \;=\; 
\frac{\Omega_{pi}(\hat{\bf k})\sigma^0_{ij}({\bf k}) \hat{k}_j \hat{k}_q + 
      \Omega_{qi}(\hat{\bf k})\sigma^0_{ij}({\bf k}) \hat{k}_j \hat{k}_p}
 {2},
\end{equation}
\begin{equation}
 {\epsilon}_{pq}({\bf x}) \;=\; \bar{\epsilon}_{pq} + 
 \tilde{\epsilon}_{pq}({\bf x}), \;\;\;\;
 \int d^3{\bf x} \tilde{\epsilon}_{pq}({\bf x}) = 0.
\end{equation}
The rotation field ${\bf R}({\bf x})={\bf I}+{\bf W}({\bf x})$ 
that corresponds to
(\ref{DisplacementHomogeneousSolution}) displacement field is
\begin{equation}
 W_{pq}({\bf k}) \;=\; 
\frac{\Omega_{pi}(\hat{\bf k})\sigma^0_{ij}({\bf k}) \hat{k}_j \hat{k}_q - 
      \Omega_{qi}(\hat{\bf k})\sigma^0_{ij}({\bf k}) \hat{k}_j \hat{k}_p}
 {2}.
\end{equation}
The stress field is
\begin{eqnarray}
 \sigma_{ij}({\bf x}) =&& c_{ijpq}^0\bar{\epsilon}_{pq} 
+ c_{ijpq}^0\tilde{\epsilon}_{pq}({\bf x}) - \sigma^0_{ij}({\bf x})
 \label{StressHomogeneousSolutionRealSpace}
\end{eqnarray}
\begin{eqnarray}
 \sigma_{ij}({\bf k}) =&& \det|{\bf H}_0|c_{ijpq}^0\bar{\epsilon}_{pq} 
\delta_{\bf k}+
c_{ijpq}^0\tilde{\epsilon}_{pq}({\bf k}) 
- \sigma^0_{ij}({\bf k}).
  \label{StressHomogeneousSolutionkSpace}
\end{eqnarray}

(\ref{ElasticEnergyDefinition}) is then relaxed to be:
\begin{eqnarray}
F^{\rm el}[{\bf H}, \bmath{\epsilon}^0({\bf x})] =&& 
 \int \frac{d^3{\bf x}}{2} c^0_{ijpq}
 \epsilon^0_{ij}({\bf x})  
 \epsilon^0_{pq}({\bf x}) - 
\int d^3{\bf x} c^0_{ijpq} \epsilon^0_{ij}({\bf x}) {\epsilon}_{pq}({\bf x})
+ 
\int \frac{d^3{\bf x}}{2} c^0_{ijpq} 
{\epsilon}_{ij}({\bf x}){\epsilon}_{pq}({\bf x})
 \nonumber\\
=&& 
 \int \frac{d^3{\bf x}}{2} c^0_{ijpq}
 \epsilon^0_{ij}({\bf x})  
 \epsilon^0_{pq}({\bf x}) - 
\int d^3{\bf x} c^0_{ijpq} \epsilon^0_{ij}({\bf x}) \bar{\epsilon}_{pq}
+ 
\int \frac{d^3{\bf x}}{2} c^0_{ijpq} \bar{\epsilon}_{ij} \bar{\epsilon}_{pq}
 \nonumber\\
-&& \int d^3{\bf x} c^0_{ijpq} \epsilon^0_{ij}({\bf x})
  \tilde{\epsilon}_{pq}({\bf x}) + 
\int \frac{d^3{\bf x}}{2} c^0_{p'q'pq} \tilde{\epsilon}_{p'q'}({\bf x}) 
\tilde{\epsilon}_{pq}({\bf x}) 
\nonumber\\
=&& 
 \int \frac{d^3{\bf x}}{2} c^0_{ijpq}
 \epsilon^0_{ij}({\bf x})  
 \epsilon^0_{pq}({\bf x}) - 
\bar{\epsilon}_{pq}\int d^3{\bf x} c^0_{ijpq} \epsilon^0_{ij}({\bf x}) 
+ 
\frac{\det|{\bf H}_0|}{2} c^0_{ijpq} \bar{\epsilon}_{ij} \bar{\epsilon}_{pq}
 \nonumber\\
-&& \frac{1}{\det|{\bf H}_0|}\sum_{\bf k}
\sigma^0_{pq}({\bf k})
\Omega_{pi}(\hat{\bf k})\sigma^{0*}_{ij}({\bf k}) \hat{k}_j \hat{k}_q
 \nonumber\\
 +&&
\frac{1}{2\det|{\bf H}_0|}\sum_{\bf k}
c^0_{p'q'pq} 
\Omega_{p'i'}(\hat{\bf k})\sigma^0_{i'j'}({\bf k}) \hat{k}_{j'} \hat{k}_{q'} 
\Omega_{pi}(\hat{\bf k})\sigma^{0*}_{ij}({\bf k}) \hat{k}_j \hat{k}_q 
\end{eqnarray}

But 
\begin{eqnarray}
c^0_{p'q'pq} 
\Omega_{p'i'}(\hat{\bf k})\sigma^0_{i'j'}({\bf k}) \hat{k}_{j'} \hat{k}_{q'} 
\Omega_{pi}(\hat{\bf k})\sigma^{0*}_{ij}({\bf k}) \hat{k}_j \hat{k}_q
=&& 
 C_{p'p}(\hat{\bf k})\Omega_{p'i'}(\hat{\bf k})\sigma^0_{i'j'}({\bf k}) \hat{k}_{j'}\Omega_{pi}(\hat{\bf k})\sigma^{0*}_{ij}({\bf k}) \hat{k}_j
\nonumber\\ =&& 
 \delta_{i'p} \sigma^0_{i'j'}({\bf k}) \hat{k}_{j'}\Omega_{pi}(\hat{\bf k})\sigma^{0*}_{ij}({\bf k}) \hat{k}_j
\nonumber\\ =&& 
  \sigma^0_{pj'}({\bf k}) \hat{k}_{j'}\Omega_{pi}(\hat{\bf k})\sigma^{0*}_{ij}({\bf k}) \hat{k}_j
\nonumber\\ =&& 
  \hat{k}_{j'} \sigma^0_{j'p}({\bf k}) \Omega_{pi}(\hat{\bf k})\sigma^{0*}_{ij}({\bf k}) \hat{k}_j.
\end{eqnarray}

So the final relaxed elastic energy \cite{WangJK02} is 
\begin{eqnarray}
F^{\rm el}[\bar{\bmath{\epsilon}}, \bmath{\epsilon}^0({\bf x})] =&& 
 \int \frac{d^3{\bf x}}{2} c^0_{ijpq}
 \epsilon^0_{ij}({\bf x})  
 \epsilon^0_{pq}({\bf x}) - 
\bar{\epsilon}_{pq}\int d^3{\bf x} c^0_{ijpq} \epsilon^0_{ij}({\bf x}) 
+ 
\frac{\det|{\bf H}_0|}{2} c^0_{ijpq} \bar{\epsilon}_{ij} \bar{\epsilon}_{pq}
 \nonumber\\
-&& \frac{1}{2\det|{\bf H}_0|}\sum_{\bf k}
\hat{k}_q \sigma^0_{qp}({\bf k})
\Omega_{pi}(\hat{\bf k})\sigma^{0*}_{ij}({\bf k}) \hat{k}_j.
  \label{TotalElasticEnergyHomogeneousSolution}
\end{eqnarray}

The supercell stress $\bar{\bmath{\sigma}}$ is 
\begin{eqnarray}
 \bar{\sigma}_{ij} \;\equiv&& \frac{1}{\det|{\bf H}_0|}
 \left.\frac{\partial F^{\rm el}[\bar{\bmath{\epsilon}},
 \bmath{\epsilon}^0({\bf x})]}{\partial
 \bar{\epsilon}_{ij}}\right|_{\bmath{\epsilon}^0({\bf x})} \nonumber\\
=&&
 c^0_{ijpq} \bar{\epsilon}_{pq} - \frac{1}{\det|{\bf H}_0|} \int d^3{\bf x} c^0_{ijpq} \epsilon^0_{pq}({\bf x})\nonumber\\
=&&
 \frac{1}{\det|{\bf H}_0|} \int d^3{\bf x} 
c^0_{ijpq} \epsilon_{pq}({\bf x}) - \frac{1}{\det|{\bf H}_0|} \int d^3{\bf x} c^0_{ijpq} \epsilon^0_{pq}({\bf x}) \nonumber\\
 =&& \frac{1}{\det|{\bf H}_0|} \int d^3{\bf x} \sigma_{ij}({\bf x}),
 \label{SupercellVirialStressHomogeneousSolution}
\end{eqnarray}
which is physically intuitive.


\subsection{General Solver}

Wang, Jin and Khachaturyan (WJK) proposed an iterative solver to
(\ref{StressEquilibrium}) based on an operator splitting technique.
The idea is one wants to avoid direct handling of ${\bf u}({\bf x})$,
and real-space representations of $\nabla^2$-like operators, as in the
usual finite-difference scheme. The finite-difference or
finite-element schemes are philosophically similar to atomistic
simulations.  It is known that solving elasticity problems in real
space often have slow convergence. In the WJK treatment, the section
\ref{sec:HomogeneousSpecialCase} solver is used as a
``pre-conditioner''.  If the system is close to an elastically
homogeneous state, the inhomogeneity can be regarded as a perturbation
and convergence should be fast.

The key idea in \cite{WangJK02} is the introduction of a reference
homogeneous system $c^\circ_{ijpq}$, which has the same displacement
field ${\bf u}({\bf x})$, strain field $\bmath{\epsilon}({\bf x})$ and
stress field $\bmath{\sigma}({\bf x})$ as the real inhomogeneous
system.  This can always be done by tuning the virtual stress free
strain field $\epsilon^\circ_{pq}({\bf x})$:
\begin{equation}
 c_{ijpq}({\bf x}) ( u_{p,q}({\bf x}) - \epsilon^0_{pq}({\bf x})) \;=\; 
 c^\circ_{ijpq} (u_{p,q}({\bf x}) - \epsilon^\circ_{pq}({\bf x})).
 \label{StressMapping}
\end{equation}
where there are as many equations (stress components) as unknowns
(stress free strain components, which do not need to satisfy
compatibility \cite{Jagla07}), and have unique solution with positive
definite $c^\circ_{ijpq}$.  So there is one-to-one mapping between a
given inhomogeneous system to a virtual homogeneous system, and vice
versa, similar to the mapping from interacting-electrons system to
non-interacting-electrons system in density functional theory (DFT)
\cite{KohnS65}.  In hindsight, the successes of the Kohn-Sham
treatment of DFT and planewave solvers (in contrast to older
Thomas-Fermi treatment, which forced to be completely local) largely
originated from the splitting of the kinetic energy ($\nabla^2$
operator which has nonlocal effects, such as boundary sensitivity)
from the total energy, Eq. (2) in \cite{KohnS65}.  The remainder part,
defined as exchange-correlation energy, is more local.  The WJK
treatment which takes advantage of planewave solver for virtual
homogeneous system is quite similar to Kohn-Sham treatment
philosophically.

Suppose {\em we know} what $\bmath{\epsilon}^\circ({\bf x})$ should be
used, it is easy to obtain ${\bf u}({\bf x})$, $\bmath{\epsilon}({\bf
x})$ and $\bmath{\sigma}({\bf x})$ based on section
\ref{sec:HomogeneousSpecialCase} nonlocal solver:
$\bmath{\epsilon}^\circ({\bf x}) \rightarrow {\bf u}({\bf
x}),\bmath{\epsilon}({\bf x}),\bmath{\sigma}({\bf x})$. This set of
$\bmath{\epsilon}({\bf x}),\bmath{\sigma}({\bf x})$ is supposed to be
identical as that of the inhomogeneous system.  But, is is true?  We
can plug into (\ref{StressMapping}) {\em locally} and check:
\begin{equation}
 c^\circ_{ijpq} \epsilon^\circ_{pq}({\bf x}) \;=\; 
 c_{ijpq}({\bf x}) \epsilon^0_{pq}({\bf x}) + 
 (c^\circ_{ijpq} - c_{ijpq}({\bf x})) \epsilon_{pq}({\bf x}).
 \label{RealSpaceInversion}
\end{equation}
The above should be satisfied exactly if we have exact guess for
$\bmath{\epsilon}^\circ({\bf x})$.  But if our estimate of
$\bmath{\epsilon}^\circ({\bf x})$ contains some error, the LHS will
not be exactly the same as the RHS.  But then we can invert the RHS to
update the guess $\epsilon^\circ_{pq}({\bf x})$, and repeat the
process until convergence is reached.

When convergence is reached, we have from (\ref{ElasticEnergyDefinition})
\begin{eqnarray}
 F^{\rm el}[\bar{\bmath{\epsilon}}, \bmath{\epsilon}^0({\bf x})] =&&
 \frac{1}{2}\int d^3{\bf x}  \sigma_{pq}({\bf x})
 ( \epsilon_{pq}({\bf x}) - \epsilon^\circ_{pq}({\bf x}) 
                          + \epsilon^\circ_{pq}({\bf x}) 
                          - \epsilon^0_{pq}({\bf x}) ) \nonumber\\
 =&& F^{\rm el\circ}[\bar{\bmath{\epsilon}}, 
\bmath{\epsilon}^\circ({\bf x})] + 
 \int \frac{d^3{\bf x}}{2} \sigma_{pq}({\bf x}) 
 (\epsilon^\circ_{pq}({\bf x}) - \epsilon^0_{pq}({\bf x})).
   \label{TotalElasticEnergyInhomogeneousSolution}
\end{eqnarray}
So the mapping of energy needs a correction.

The supercell stress $\bar{\bmath{\sigma}}$ is 
\begin{eqnarray}
 \bar{\sigma}_{ij} \;\equiv&& \frac{1}{\det|{\bf H}_0|}
 \left.\frac{\partial F^{\rm el}[\bar{\bmath{\epsilon}},
 \bmath{\epsilon}^0({\bf x})]}{\partial
 \bar{\epsilon}_{ij}}\right|_{\bmath{\epsilon}^0({\bf x})} \nonumber\\
=&& \frac{1}{\det|{\bf H}_0|}
 \left.\frac{\partial F^{\rm el\circ}[\bar{\bmath{\epsilon}},
 \bmath{\epsilon}^0({\bf x})]}{\partial
 \bar{\epsilon}_{ij}}\right|_{\bmath{\epsilon}^0({\bf x})}.
\end{eqnarray}
The reason is that in (\ref{TotalElasticEnergyInhomogeneousSolution}),
the value of $F^{\rm el}$ obviously depends parametrically on
$\bmath{\epsilon}^\circ({\bf x})$, and with change in
$\bar{\bmath{\epsilon}}$ there will be associated
$\delta\bmath{\epsilon}^\circ({\bf x})$.  However, 
\begin{equation}
 \frac{\delta F^{\rm el}[\bmath{\epsilon}^\circ({\bf x}) | \bar{\bmath{\epsilon}}, \bmath{\epsilon}^0({\bf x})] }{\delta \bmath{\epsilon}^\circ({\bf x})} \;=\; 0
\end{equation}
so (\ref{SupercellVirialStressHomogeneousSolution}) can be used, which
is physically intuitive.


\subsection{3D Isotropic Media}
\label{sec:3DIsotropicMedia}

A 3D isotropic medium has 
\begin{equation}
 c_{ijpq} \;=\; \lambda \delta_{ij}\delta_{pq} + 2\mu \delta_{ip}\delta_{jq}
\end{equation}
The relationship between the Lam\'{e} parameters $\lambda,\mu$ and $E,\nu$ are:
\begin{equation}
 \lambda = \frac{2\nu\mu}{1-2\nu} = \frac{E\nu}{(1+\nu)(1-2\nu)}, \;\;\;
 \mu = \frac{E}{2(1+\nu)},
\end{equation}
and the relationship between stress and strain is:
\begin{equation}
 \sigma^0_{ij}({\bf k}) = (\lambda\epsilon^0_{pp}({\bf k}))\delta_{ij} 
                            + 2\mu\epsilon^0_{ij}({\bf k}), \;\;\;
 \sigma^0_{ij}({\bf x}) = (\lambda\epsilon^0_{pp}({\bf x}))\delta_{ij} 
                            + 2\mu\epsilon^0_{ij}({\bf x}).
\end{equation}

Then (\ref{C_Definition}) becomes:
\begin{equation}
 C_{ip}(\hat{\bf k}) = \lambda \hat{k}_i \hat{k}_p + 2\mu \delta_{ip}
\end{equation}
or ${\bf C}(\hat{\bf k})=2\mu{\bf I}+\lambda \hat{\bf K}$ with
idempotent $\hat{\bf K}$ matrix:  $\hat{\bf K}^n=\hat{\bf K}$.

The inversion of ${\bf C}(\hat{\bf k})$ can be done by matrix series expansion:
\begin{equation}
 {\bf \Omega}(\hat{\bf k}) \;=\; 
\frac{1}{2\mu}\sum_{n=0}^\infty (-\frac{\lambda}{2\mu})^n \hat{\bf K}^n 
= \frac{1}{2\mu} ( {\bf I} - \frac{\lambda}{2\mu} \frac{\hat{\bf K}}{1+\frac{\lambda}{2\mu}} ) = \frac{1}{2\mu} ( {\bf I} - \frac{\lambda\hat{\bf K}}{\lambda+2\mu} ).
\end{equation}
Define dimensionless quantity
\begin{equation}
 \alpha \;\equiv\; \frac{\lambda}{\lambda+2\mu} = \frac{\nu}{1-\nu},
\end{equation}
we then have ${\bf \Omega}(\hat{\bf k}) = ({\bf I} - \alpha\hat{\bf K})/2\mu$.

So (\ref{DisplacementHomogeneousSolution}) would become
\begin{equation}
 \tilde{u}_p({\bf k}) \;=\; 
 \frac{(\delta_{pi}-\alpha\hat{k}_p\hat{k}_i)\sigma^0_{ij}({\bf k}) \hat{k}_j}
 {2\mu i|{\bf k}|} = \frac{\sigma^0_{pj}({\bf k}) \hat{k}_j - 
 \alpha\hat{k}_p\sigma^0_{ij}({\bf k})\hat{k}_i\hat{k}_j}
 {2\mu i|{\bf k}|}.
\end{equation}
Define vector and scalar 
\begin{equation}
 {\bf f}({\bf k}) \;\equiv\; \bmath{\sigma}^0({\bf k})\cdot \hat{\bf k}, \;\;\;
 g({\bf k}) \;\equiv\; \hat{\bf k}\cdot {\bf f}({\bf k}),
\end{equation}
which can be pre-computed, we then have
\begin{equation}
 \tilde{\bf u}({\bf k}) \;=\; \frac{{\bf f}({\bf k}) - \alpha g({\bf k})\hat{\bf k}}{2\mu i|{\bf k}|}.
\end{equation}

The periodic part of the strain field is then
\begin{equation}
 \tilde{\bmath{\epsilon}}({\bf k}) \;=\; 
\frac{{\bf f}({\bf k})\hat{\bf k}+\hat{\bf k}{\bf f}({\bf k}) - 2\alpha g({\bf k})\hat{\bf K}}
 {4\mu},
\end{equation}
with ${\rm tr}({\bf f}({\bf k})\hat{\bf k})={\rm tr}(\hat{\bf k}{\bf
f}({\bf k}))=g({\bf k})$, ${\rm tr}(\tilde{\bmath{\epsilon}}({\bf k}))=(1-\alpha)g({\bf k})/2\mu$, and
\begin{equation}
 \bmath{\epsilon}({\bf x}) \;=\; \bar{\bmath{\epsilon}} + 
 \tilde{\bmath{\epsilon}}({\bf x}), \;\;\;\;
 \int d^3{\bf x} \tilde{\bmath{\epsilon}}({\bf x}) = 0.
\end{equation}
The rotation field ${\bf R}({\bf x})={\bf I}+{\bf W}({\bf x})$ 
 field is
\begin{equation}
 {\bf W}({\bf k}) \;=\; 
\frac{{\bf f}({\bf k})\hat{\bf k}-\hat{\bf k}{\bf f}({\bf k})}
 {4\mu}.
\end{equation}
The $c_{ijpq}^0\tilde{\epsilon}_{pq}({\bf k})$
stress field in (\ref{StressHomogeneousSolutionkSpace}) is simplified to be
\begin{eqnarray}
 \lambda {\rm tr}(\tilde{\bmath{\epsilon}}({\bf k})){\bf I} + 2\mu \tilde{\bmath{\epsilon}}({\bf k})  =&&  \frac{\lambda(1-\alpha)g({\bf k}){\bf I}}{2\mu} + 
 \frac{{\bf f}({\bf k})\hat{\bf k}+\hat{\bf k}{\bf f}({\bf k}) - 2\alpha g({\bf k})\hat{\bf K}}
 {2} \nonumber\\
 =&& \alpha g({\bf k}){\bf I} +  \frac{{\bf f}({\bf k})\hat{\bf k}+\hat{\bf k}{\bf f}({\bf k}) - 2\alpha g({\bf k})\hat{\bf K}}
 {2}\nonumber\\
 =&& \alpha g({\bf k})({\bf I}-\hat{\bf K}) +  \frac{{\bf f}({\bf k})\hat{\bf k}+\hat{\bf k}{\bf f}({\bf k})}{2}
\end{eqnarray}
so
\begin{eqnarray}
 \sigma_{ij}({\bf k}) =&& \det|{\bf H}_0|c_{ijpq}^0\bar{\epsilon}_{pq} 
\delta_{\bf k}+
 \alpha g({\bf k})({\bf I}-\hat{\bf K}) +  \frac{{\bf f}({\bf k})\hat{\bf k}+\hat{\bf k}{\bf f}({\bf k})}{2}
- \sigma^0_{ij}({\bf k}).
  \label{StressIsotropicHomogeneousSolutionkSpace}
\end{eqnarray}

In the real-space inversion of (\ref{RealSpaceInversion}):
\begin{equation}
 c^\circ_{ijpq} \epsilon^\circ_{pq}({\bf x}) 
 \;=\; \tau_{ij}({\bf x}), \;\;\;\;
 \lambda{\rm tr}(\bmath{\epsilon}^\circ){\bf I} + 
 2\mu\bmath{\epsilon}^\circ \;=\; \bmath{\tau},
\end{equation}
we note that
\begin{equation}
 3\lambda{\rm tr}(\bmath{\epsilon}^\circ) + 
 2\mu {\rm tr}(\bmath{\epsilon}^\circ)
\;=\; {\rm tr}(\bmath{\tau}), \;\;\;\;
 {\rm tr}(\bmath{\epsilon}^\circ) \;=\; 
 \frac{{\rm tr}(\bmath{\tau})}{3\lambda+2\mu},
\end{equation}
so 
\begin{equation}
  \bmath{\epsilon}^\circ \;=\; \frac{\bmath{\tau}}{2\mu} - 
  \frac{\lambda}{2\mu}
 \frac{{\rm tr}(\bmath{\tau})}{3\lambda+2\mu} {\bf I}.
\end{equation}

\bibliography{MyBibliography}
\end{document}
