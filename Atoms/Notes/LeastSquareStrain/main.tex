\documentclass[12pt]{article}
\usepackage{natbib,graphicx,fullpage,doublespace,amssymb,amsfonts}
\setlength{\parindent}{0em}
\setlength{\parskip}{0.65em}
\setstretch{1.15}
\def\bmath#1{\mbox{\boldmath$#1$}}

% \documentclass[aps,prb,preprint,showpacs,amsmath,amssymb]{revtex4} %
% \documentclass[aps,prb,amsmath,amssymb]{revtex4} %
% \usepackage{graphicx,subfigure,bm} %

\usepackage{bm,yhmath}
\usepackage{url}

\begin{document}
\bibliographystyle{unsrt}

\title{Least-Square Atomic Strain}

\author{Ju Li, Futoshi Shimizu}

\date{\today}

\maketitle

Since strain is by definition a relative measure\cite{WangLYPW95},
usually one needs two atomistic configurations, the reference and the
current, in order to compute the local strain. A specialized trick
could work if the reference configuration is known to be of high
symmetry, and that only provides the shear
invariant\cite{Li99,Li03}. To compute the full transformation matrix
${\bf J}$ and strain tensor $\bmath{\eta}$, one needs two
configurations.

Define integer $N_i$ to be the number of neighbors of atom $i$ in the
present configuration. For each neighbor $j$ of atom $i$, their
present separation is
\begin{equation}
 {\bf d}_{ji} \;\equiv\; {\bf x}_j - {\bf x}_i,
\end{equation}
and their old separation was
\begin{equation}
 {\bf d}_{ji}^0 \;\equiv\; {\bf x}_j^0 - {\bf x}_i^0.
\end{equation}
Note that in our present programs\cite{Li05-2.8,Li05-2.19,Li05-2.31},
${\bf d}_{ji}$ and ${\bf d}_{ji}^0$ etc. are all considered to be row
vectors.

We seek a locally affine transformation matrix ${\bf J}_i$, that best
maps
\begin{equation}
 \{{\bf d}_{ji}^0\} \;\rightarrow\; \{{\bf d}_{ji}\},  \;\;\;\;
 \forall j\in N_i
\end{equation}
in other words, we seek ${\bf J}_i$ that minimizes:
\begin{eqnarray}
\sum_{j\in N_i} | {\bf d}_{ji}^0{\bf J}_i - {\bf d}_{ji} |^2 
 \;\;=&& \sum_{j\in N_i} ({\bf d}_{ji}^0{\bf J}_i - {\bf d}_{ji}) 
 ({\bf d}_{ji}^0{\bf J}_i - {\bf d}_{ji})^T \nonumber\\
=&& \sum_{j\in N_i} {\rm Tr} (({\bf d}_{ji}^0{\bf J}_i - {\bf
 d}_{ji})^T ({\bf d}_{ji}^0{\bf J}_i - {\bf d}_{ji})) \nonumber\\
=&& {\rm Tr} \sum_{j\in N_i} ({\bf J}_i^T {\bf d}_{ji}^{0T} - {\bf d}_{ji}^T) 
({\bf d}_{ji}^0{\bf J}_i - {\bf d}_{ji}).
\end{eqnarray}

Performing arbitrary matrix variation $\delta {\bf J}_i^T$ in above,
we get
\begin{eqnarray}
 0 \;\;=&& 
{\rm Tr} \sum_{j\in N_i} \delta{\bf J}_i^T {\bf d}_{ji}^{0T} 
({\bf d}_{ji}^0{\bf J}_i - {\bf d}_{ji})  \nonumber\\
=&& {\rm Tr} \delta{\bf J}_i^T \sum_{j\in N_i} {\bf d}_{ji}^{0T} 
({\bf d}_{ji}^0{\bf J}_i - {\bf d}_{ji}).
\end{eqnarray}

For the above to be true for any $\delta {\bf J}_i^T$, the matrix
\begin{equation}
  \sum_{j\in N_i} {\bf d}_{ji}^{0T} 
({\bf d}_{ji}^0{\bf J}_i - {\bf d}_{ji})
\end{equation}
has to be zero. In other words, there must be:
\begin{equation}
  \left(\sum_{j\in N_i} {\bf d}_{ji}^{0T} {\bf d}_{ji}^0\right) 
  {\bf J}_i \;=\; \sum_{j\in N_i} {\bf d}_{ji}^{0T} {\bf d}_{ji}.
\end{equation}
If we define
\begin{equation}
 {\bf V}_i \;\equiv\; \sum_{j\in N_i} {\bf d}_{ji}^{0T} {\bf
 d}_{ji}^0, \;\;\;\; 
 {\bf W}_i \;\equiv\; \sum_{j\in N_i} {\bf d}_{ji}^{0T} {\bf
 d}_{ji},
\end{equation}
then there is simply
\begin{equation}
 {\bf J}_i \;=\; {\bf V}_i^{-1} {\bf W}_i
\end{equation}

The Lagrangian strain matrix is then calculated as
\begin{equation}
 \bmath{\eta}_i \;=\; \frac{1}{2}\left({\bf J}_i {\bf J}_i^T - {\bf
 I}\right).
 \label{LocalLagrangianStrain}
\end{equation}
And we can then compute its hydrostatic invariant
\begin{equation}
 \eta^{\rm hydro}_i = \frac{1}{3} {\rm Tr} \bmath{\eta}_i,
\end{equation}
calibrated to $\eta_{xx}=\eta_{yy}=\eta_{zz}=a$, and the local shear
invariant,
\begin{eqnarray}
 \eta^{\rm Mises}_i &&=\;\; \sqrt{\frac{1}{2} 
{\rm Tr} ( \eta - \eta_m {\bf I} )^2} \nonumber \\
&&=\;\; \sqrt{\eta_{yz}^2+\eta_{xz}^2+\eta_{xy}^2+
\frac{(\eta_{yy}-\eta_{zz})^2+(\eta_{xx}-\eta_{zz})^2+(\eta_{xx}-\eta_{yy})^2}
{6} }.
\end{eqnarray}
calibrated to $\eta_{xy}=\eta_{yx}=b$, and all others zero.

Note that
\begin{enumerate}
 \item The order of ${\bf J}_i$ and ${\bf J}_i^T$ in
 (\ref{LocalLagrangianStrain}) matrix product is important. The wrong
 product order would also give you a symmetric matrix, but that
 symmetric matrix has nothing to do with the strain.
 \item The product order of (\ref{LocalLagrangianStrain}) {\em
 appears} to be the reverse of that of \cite{WangLYPW95}, because we
 adopt row-based vector scheme now
 \cite{Li05-2.8,Li05-2.19,Li05-2.31}, while column-based scheme was
 adopted in \cite{WangLYPW95}.
\end{enumerate}

\bibliography{MyBibliography}

\end{document}
