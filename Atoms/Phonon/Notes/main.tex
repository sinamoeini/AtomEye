\documentclass[12pt]{article}
\usepackage{fullpage,doublespace,graphicx}
\setlength{\parindent}{0em}
\setlength{\parskip}{0.35em}
\setstretch{1.5}

\title{On Temperature Rescaling}
\date{\today}
\author{Ju Li}
\begin{document}
\maketitle

Debye proposed the following single-parameter spectrum density,
\begin{equation}
dP = d\left (\frac{\omega}{\omega_{\rm D}} \right)^3, 
\;\;\omega < \omega_{\rm D}, \;\;\;\;\;\;0,\;\;\omega \ge \omega_{\rm
D}, \label{Debye_DOS}
\end{equation}
where the normalization is a single {\em degree of freedom}, which
{\em should} possess $k_{\rm B}T$ total energy under classical
mechanics and harmonic approximation. In contrast, under quantum
mechanics, the total energy is
\begin{equation}
\langle E \rangle = \int_0^{\omega_{\rm D}} 
d\left (\frac{\omega}{\omega_{\rm D}} \right)^3
\left ( \frac{1}{2} + 
\frac{1}{\exp\left ( \frac{\hbar\omega}{k_{\rm B}T} \right ) - 1} \right )
\hbar \omega. 
\end{equation}

Let us define
\begin{equation}
k_{\rm B}T_{\rm D} \;\;\equiv\;\; \hbar \omega_{\rm D}.
\end{equation}
We then have the quantum energy average,
\begin{equation}
\langle E \rangle = k_{\rm B} T_{\rm D}
\int_0^{\omega_{\rm D}}
d\left (\frac{\omega}{\omega_{\rm D}} \right)^3
\left ( \frac{1}{2} + \frac{1}{\exp\left ( 
\frac{T_{\rm D}}{T}\cdot \frac{\omega}{\omega_{\rm D}} 
\right ) - 1} \right )
\frac{\omega}{\omega_{\rm D}},
\end{equation}
which can be written as,
\begin{equation}
\langle E \rangle = k_{\rm B} T_{\rm D} \int_0^1
dy^3 \left ( \frac{1}{2} + 
\frac{1}{\exp\left ( \frac{T_{\rm D}}{T}\cdot y \right ) - 1} 
\right ) y,
\end{equation}
or,
\begin{equation}
\langle E \rangle = k_{\rm B} T_{\rm D} \int_0^1
dy \left ( \frac{1}{2} + 
\frac{1}{\exp\left ( \frac{T_{\rm D}}{T}\cdot y \right ) - 1} \right ) 
3y^3.
\label{harmonic_total_energy}
\end{equation}

Alternatively, one can rewrite (\ref{harmonic_total_energy}) as
\begin{equation}
\langle E \rangle = k_{\rm B} T_{\rm D} 
\left ( \frac{T}{T_{\rm D}} \right )^4
\int_0^{\frac{T_{\rm D}}{T}}
dy \left ( \frac{1}{2} +
\frac{1}{e^y - 1} \right ) 3y^3.
\end{equation}

Therefore, if we require the classical system to have equal energy as
the quantum system on average, we would demand
\begin{equation}
T_{\rm MD} = T_{\rm D} \left ( \frac{T}{T_{\rm D}} \right )^4
\int_0^{\frac{T_{\rm D}}{T}} dy \left ( \frac{1}{2} +
\frac{1}{e^y - 1} \right ) 3y^3.
\label{Debye_temperature_rescaling}
\end{equation}
This form is more ready for numerical evaluation.  We see that when
$T\rightarrow 0$, $T_{\rm MD}\rightarrow (3/8)T_{\rm D}$, a nonzero
value. But when $T\rightarrow \infty$, $T_{\rm MD} = T + {\cal
O}(1/T)$.

To get the heat capacity, or $dT_{\rm MD}/dT$, it is easier to work on
(\ref{harmonic_total_energy}):
\begin{equation}
\frac{dT_{\rm MD}}{dT}
 = \left ( \frac{T_{\rm D}}{T} \right )^2
\int_0^1
dy \frac{
\exp\left ( \frac{T_{\rm D}}{T}\cdot y \right ) }{\left (
\exp\left ( \frac{T_{\rm D}}{T}\cdot y \right ) - 1 \right )^2 }
3y^4,
\end{equation}
or alternatively,
\begin{equation}
\frac{dT_{\rm MD}}{dT}
 = \left ( \frac{T}{T_{\rm D}} \right )^3
\int_0^{\frac{T_{\rm D}}{T}}
dy \frac{3y^4 e^y}{\left (e^y- 1\right )^2} \equiv
D\left( \frac{T}{T_{\rm D}} \right ),
\label{Debye_heat_capacity}
\end{equation}
where $D(x)$ is the well-known Debye function \cite{McQuarrie76}, with
$D(x)\sim (4\pi^4/5)x^3$ as $x\sim 0$ and $D(x)\rightarrow 1$ as
$x\rightarrow \infty$.

In practice, the phonon spectrum density or DOS is of course {\em not}
in the same form as (\ref{Debye_DOS}). Especially, multi-component
systems have optical bands that are of entirely different structure
than (\ref{Debye_DOS}). Nevertheless,
(\ref{Debye_temperature_rescaling}) and (\ref{Debye_heat_capacity})
provide good functional forms for $T_{\rm MD}(T)$ representation, both
numerically and physically. Let us consider the temperature rescaling
procedure's physical effect on classical MD: it can be proved that the
initial $\hbar\omega/2$ and the later flatter $T_{\rm MD}(T)$
excitation in classical MD mimics the effect of quantum phonon
Boltzmann equation, but with the unsatisfactory aspect that this
correction is incorrectly smeared out throughout the entire spectrum
(if one allows the classical system to equilibrate), so only an
average correction effect remains.  Given that, the question is
suppose we have a wide spectrum which contains optical band that is
far from the acoustic band, {\em which part should one sacrifice},
that is, {\em more} incorrectly smeared out?  It is not hard to see
that it should be the optical bands, since they do not contribute
significantly to the thermal conductivity. Furthermore, one often
worries more about the quantum effects on thermo-mechanical properties
near room temperature, and less for when the temperature is high, say
$T=1000$K. The reason is not just because these properties might be
less important at high temperature than at room temperature, but also
because the quantum effects themselves are not significant enough to
have a big influence at high temperature. If we think about it, it is
common practice to do classical MD without {\em any} correction - but
one is still expected to get sound result at high temperature, so it
must be all right if we apply correction but {\em which is not neatly
fitted for high temperature}. Therefore, I think it is a good idea to
fit (\ref{Debye_heat_capacity}) to the actual heat capacity at
$T=300$K, where only the acoustic band is likely to be activated.

\begin{thebibliography}{00}
\begin{singlespace}
\bibitem{McQuarrie76} {\em Statistical Mechanics}, D. A. McQuarrie,
Harper Collins Publishers (New York, 1976).
\end{singlespace}
\end{thebibliography}

\end{document}
